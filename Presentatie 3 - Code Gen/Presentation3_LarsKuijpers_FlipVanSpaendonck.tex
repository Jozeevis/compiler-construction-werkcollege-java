\documentclass{beamer}
\usepackage{ru}
\author[Flip \& Lars]{Flip van Spaendonck \& Lars Kuijpers}
\title{Compiler Construction \\ \lbrack VSS\rbrack : I'm building stacks }
%\setbeamercovered{transparent} 
%\setbeamertemplate{navigation symbols}{} 
%\logo{} 
%\institute{} 
%\date{} 
%\subject{} 
\begin{document}

\begin{frame}
\titlepage
\end{frame}

\begin{frame}
\frametitle{Semantics}
\begin{itemize}
\item Call-by-reference
\item Print only prints basic types (others as pointers)
\end{itemize}
% examples (tuple)
\end{frame}

\begin{frame}
\frametitle{Code generation}
\begin{itemize}
\item Each token has an addCodeToStack(stack, counter) function
\item Call this for tree node
\item Each node asks his children to add their code to the codestack
\item Counter used to make unique labels
\end{itemize}
\end{frame}

\begin{frame}
\frametitle{Compilation Schemes (1/2)}
\textbf{Function calls}
\begin{itemize}
\item Give a unique name to each function to use as label
\item Put arguments' code on the codestack 
\item Results will be put as variables on memory stack
\item Jump to code for function body (save origin point to stack)
\item Return will set PC back to origin point
\item Result of function (if any) will be on top of stack
\end{itemize}
\end{frame}

\begin{frame}
\frametitle{Compilation Schemes (2/2)}
\textbf{Tuples}
\begin{itemize}
\item Two adjacent addresses
\item Left is pointer address -1, right is pointer address
\end{itemize}

\textbf{Lists}
\begin{itemize}
\item Nested tuples
\item Left is pointer to the variable in the current position
\item Right is pointer to the next tuple in the list
\item Empty list is 0
\end{itemize}
\end{frame}

\begin{frame}
\frametitle{Extensions}
\textbf{Memory management}
\begin{itemize}
\item Garbage collection on function exit
\item Efficient use of registers
\end{itemize}
\textbf{Higher-order functions}
\begin{itemize}
\item How to pass functions as argument?
\end{itemize}
\textbf{Custom structures}
\begin{itemize}
\item Multuples instead of tuples
\item Structs
\end{itemize}
\end{frame}

\begin{frame}
\begin{block}{Fun Metrics}
\begin{tabular}{l || l | l}
\textbf{LOC} & 3616 & a\\
\textbf{Word count} & 9684 & a\\
\textbf{Character count} & 98470 & aaa \\
\textbf{Slides made} & 17 & 25\\
\textbf{Coffee consumed} & 0 & Still 0\\
\textbf{Sanity lost} & A bunch & Even more \\
\textbf{Experience} & Priceless & Pricelesser
\end{tabular}
\end{block}
\end{frame}

\begin{frame}
\center{\Huge{Questions ?}}
\end{frame}

\end{document}