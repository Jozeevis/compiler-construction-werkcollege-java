\documentclass{beamer}
\usepackage{ru}
\author[Flip \& Lars]{Flip van Spaendonck \& Lars Kuijpers}
\title{Compiler Construction \\ Overextending with the bois }
%\setbeamercovered{transparent} 
%\setbeamertemplate{navigation symbols}{} 
%\logo{} 
%\institute{} 
%\date{} 
%\subject{} 
\begin{document}

\begin{frame}
\titlepage
\end{frame}

\begin{frame}
\frametitle{Extension idea}
\begin{itemize}
\item More complex data structures
\item Makes it easier for the user
\item More understandable programs
\item We added 2 aspects to SPL:
\begin{itemize}
\item Muples
\item Structs
\end{itemize}
\end{itemize}
\end{frame}

\begin{frame}
\frametitle{Muples}
\begin{itemize}
\item Muples are tuples with more than two items
\item Can be arbitrarily large (within memory constraints)
\item Size is known at compile-time
\item The Nth item of a muple can be accessed by using muple.[N]
\end{itemize}
\end{frame}

\begin{frame}
\frametitle{Structs}
\begin{itemize}
\item Similar to classes in Java
\item Structs are user-defined
\item Structs have variables and functions
\item Both can be called via the struct reference
\item Every struct has to have at least one constructor
\end{itemize}
\end{frame}

\begin{frame}
\frametitle{Implementation}
\begin{itemize}
\item Muples are simply tuples with more memory space allocated to them
\item Struct variables are similar to muples, but are accessed with an id instead of a number
\item Struct functions are similar to normal functions, but they use a different local memory
\item To do this, we introduced an idea of domain to our compiler
\end{itemize}
\end{frame}

\begin{frame}
\frametitle{Domain}
\begin{itemize}
\item Domains are blocks of variable and function definitions
\item Everything will normally be in the global domain
\begin{itemize}
\item This includes struct declarations, normal functions, global variables
\end{itemize}
\item Struct functions will be executed in the domain of their struct
\end{itemize}
\end{frame}

\begin{frame}
\frametitle{Difficulties}
\begin{itemize}
\item Implementing the domains
\item Finalizing the rest of the code
\end{itemize}
\end{frame}

\begin{frame}
\begin{block}{Fun Metrics}
\begin{tabular}{l || l | l | l}
\textbf{LOC} & 3616 & 4474 & LOC is a bad metric\\
\textbf{Word count} & 9684 & 12396 & Just like this\\
\textbf{Character count} & 98470 & 124002 & Except for this one\\
\textbf{Slides made} & 17 & 25 & 34\\
\textbf{Coffee consumed} & 0 & Still 0 & We just don't drink coffee\\
\textbf{Sanity lost} & A bunch & Even more & Unhinged\\
\textbf{Experience} & Priceless & Priceless*2 & Mastercard
\end{tabular}
\end{block}
\end{frame}

\begin{frame}
\center{\Huge{Questions ?}}
\end{frame}

\end{document}